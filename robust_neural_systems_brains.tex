% !TeX spellcheck = en_GB

\section{\protect{\emoji{brain}} \textit{(relatively)} Robust neural systems: \textit{brains}}{

    \begin{frame}{\protect{\emoji{brain}} Adversarial attacks and the brain}

        \begin{center}
            \underline{Focus on vision}
        \end{center}

        It is safe to say that examples similar in form to \textit{adversarial attacks} for \textit{NNs} are \alert{yet to be discovered} for \textit{e.g.} human subjects.

        Some related phenomena, however, do \textit{probably} exist (within \textit{vision}):
        \begin{itemize}
            \item \alert{Retinal} response-elicitation (\textit{e.g. impossible colours}; incomplete evidence);
            \item \alert{Attention} retargeting via saliency shaping;
            \item Amygdala-mediated (\textit{\alert{semantic}}) attention retargeting;
            \item \alert{Fast}, \textit{adversarial} elicitation of the \alert{early} visual pathway (anecdotal).
        \end{itemize}

    ...and up to \textit{optical illusions}, with a stretch.
    \end{frame}

    \begin{frame}{\protect{\emoji{brain}} Recall, introspection... \textit{robust AI}?}

        Yet, \textit{brains} may be the \textit{only} practical realisation of a system with the \textit{robustness} properties we look for...

        \begin{block}{\protect{\emoji{light-bulb}} A guiding idea}
            Is it possible to loosely inform the development of \textit{robust} DL systems with (grossly simplified, idealised) descriptions of \alert{neurocognitive} phenomena?
        \end{block}

        Getting inspiration from the ideas of  \textit{\alert{recall} of acquired information}, and \textit{\alert{introspection}} as \textit{thought about thought}:
        \begin{itemize}
            \item Role of \textit{recall} as a \textit{\alert{comparison} tool} for newly acquired information;
            \item Role of \textit{recall} (and the aware \textit{anticipation} of it) in learning and \textit{\alert{gap-filling}} memories\\
            $\rightarrow$ \textit{Forward testing} phenomenon\\
            $\rightarrow$ \textit{Repeated testing} phenomenon
            \item \textit{Liminality} and hippocampal dynamics.
        \end{itemize}
    \end{frame}

    \begin{frame}{\protect{\emoji{brain}} An example from... poetry: \textit{recollection in tranquillity}}
        \centerline{
            \begin{minipage}[]{0.75\textwidth}
                \begin{quote}{\textsc{W. Wordsworth} -- \textit{I Wandered Lonely as a Cloud}}
                    \textelp{} I \alert{gazed}—and gazed—but little thought\\
                    What wealth the show to me had brought:\\
                    \mbox{}\\
                    For oft, when on my couch I lie\\
                    In vacant or in pensive mood,\\
                    They flash upon that \alert{inward eye} \textelp{}\\
                    $\textit{ }$
                \end{quote}
            \end{minipage}
        }
    \end{frame}

    \begin{frame}{\protect{\emoji{brain}} A crucial remark!}
        \begin{block}{\protect{\emoji{warning}} Beware!}
            The \textit{modelling} that follows has no claim of \textit{biological plausibility} whatsoever, at this stage! This would be \textit{added value}, though -- and in interesting research direction!
        \end{block}
    \end{frame}
}
