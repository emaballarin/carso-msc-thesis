% !TeX spellcheck = en_GB

\npsection{Preliminaries}

The aim of the section that follows will be to provide the essential elements of knowledge whose use will be abundant in, and required to better appreciate, the forthcoming -- while also favouring the contextualisation of novel contributions within the broader field of \textit{deep learning}, and that of \textit{adversarial defences} specifically. As such, with no pretence of exhaustiveness, the following pages will focus chiefly on a high-level conceptual overview, along a \textit{drill-down} from the goals of \textit{artificial intelligence} down towards the very specific problem and use-case considered.


\subsection{ML $\subseteq$ AI}

\textit{Artificial Intelligence} is usually defined as the field that studies tools and methods capable of reproducing higher-level cognitive functions. \textit{I.e.} rational and autonomous reasoning, decision-making and agency, and/or adaptation to complex or previously unseen scenarios. As such, it is an area that lies over a broad variety of disciplines and approaches: from computer science and mathematics, to psychology and philosophy.

The core gnosiological underpinning from which \textit{Machine Learning} and \textit{Deep Learning} move (called \textit{computational cognitivism}) posits that the previously mentioned goals of \textit{Artificial Intelligence} may be reached by reduction to -- though arbitrarily rich and complex -- algorithmic computation: thanks to the tools of mathematical formalisation and statistical-probabilistic reasoning as a way to quantify and operate under uncertainty.

\textit{Machine Learning} can then be defined as the set of rigorous mathematical techniques (spanning \textit{modelling}, \textit{algorithmics}, \textit{statistical/probabilistic learning theory}, optimisation) leading to the development of algorithms (called indeed \textit{learning algorithms}) to extract information from \textit{experience} (provided in the form of \textit{data}) without being explicitly programmed to execute the specific task \textit{learned}\footnote{Such popular definition of \textit{ML} comes from a rephrased quote from \cite{Samuel1959MachineLearning} -- whose author also helped the development of \TeX, the typesetting system which this thesis has been composed with.}.

In a more measurable fashion\footnote{Such definition is attributed to Tom Mitchell.}:

\begin{displayquote}
    \textit{A computer program is said to learn from experience E \wrt to some class of tasks T and performance measure P, if its performance at tasks in T, as measured by P, improves with experience E.}
\end{displayquote}

Specifically, in order to have \textit{program} show such kind of behaviour at a given task, it is necessary for the \textit{learning algorithm} itself to learn from the data a \textit{mathematical model} of the essential phenomena involved in the fulfilment of the task -- and whose use allows for its (eventually approximate) further execution, even on unseen input data. This clarification allows -- in theory -- to decouple the machine learning \textit{model} from the \textit{learning} (and potentially \textit{inference}) \textit{algorithm} used to determine its determining parameters.

\subsubsection{Minimal systematics}
At this point, one can preliminarily classify the (extremely varied, and often not clear-cut) landscape of machine learning tasks (and related algorithms) according to the amount of \textit{supervision} required by the learning, or \wrt to the probability distribution the \textit{model} is tasked to learn.

The former discrimination allows us to define:
\begin{itemize}
    \item \textit{Supervised} learning tasks, requiring an input/output mapping to be learned from a \textit{training} dataset of knowingly-correct pairs of the same kind (\textit{e.g.} image classification);
    \item \textit{Unsupervised} learning, where data are provided as input and properties or transformations of them are required to be learned without further exemplification of the output (\textit{e.g.} dimensionality reduction);
    \item \textit{Reinforcement learning}, involving the determination of the optimal actions (among many) to be taken according to the state the \textit{agent} and the \textit{environment} are -- by utilising only a \textit{reward function} and eventually under the further constraint of partial state observability and outcome nondeterminism.
\end{itemize}

Whereas, according to the latter description, we can have:
\begin{itemize}
    \item \textit{Generative} learning, dedicated to the modelling of the full data-generating distribution, \textit{i.e.} $p(x)$ in the case of inputs only, or the joint $p(x,y)$ in the case of input/output pairs -- or some property or transformation of them;
    \item \textit{Discriminative} learning, dealing with the less informative conditional model of $p(y|x)$ -- or some statistic of it -- in the case of input/output pairs.
\end{itemize}

\subsection{Deep Learning}

Given the above, one can simply define \textit{deep learning} as a subset of \textit{machine learning} -- whose models are \textit{deep artificial neural networks} (whose precise nature will be right introduced).

\subsubsection{From artificial neurons...}

The essential building block of an \textit{artificial neural network} -- being it \textit{shallow} or \textit{deep} -- is the \textit{artificial neuron}. Though not a proper model of its biological counterpart -- unless at a very high, conceptual level -- its structure was loosely inspired\footnote{See \cite{McCullochPitts1990ALC} and \cite{Rosenblatt1958ThePA}.} by the dendritic and axonal connectivity of neurons in a biological brain, abstracting away the differentiations that may occur.

From a mathematical modelling viewpoint, an \textit{artificial neuron} is just an affine vector-to-scalar transformation followed by a (usually, except in the trivial case) nonlinear function, called \textit{activation}.

\textit{i.e.}

$$y = \mathcal{N}_1(\vec{x}) = \mathcal{A}(b + \vec{w} \cdot \vec{x})$$

with, in the most general setting $y \in \mathbb{R}$ (the output), $\vec{x} \in \mathbb{R}^n$, $n \in \mathbb{N}_{\setminus0}$ (the vector input, also viewable as an ordered collection of scalar inputs, \textit{e.g.} the outputs of other \textit{neurons}). In such case, the activation $\mathcal{A}: \mathbb{R} \rightarrow \mathbb{R}$ and the model \textit{parameters} $b \in \mathbb{R}$ (\textit{bias}) and $\vec{w} \in \mathbb{R}^n$ (weights) -- to be learned, eventually -- fully define the model. In the usual scenario, the choice of $\mathcal{A}$ is not to be learned, but still its fixed functional form may depend on additional learnable parameters.

\subsubsection{...to deep artificial neural networks}

If we consider, at this point, a group of $N_{\ell}$ ordered (and distinct) neurons $\mathcal{N}_i$ and provide them the same vector $\vec{x}$ as input, we obtain as output $y_i = \mathcal{N}_i(\vec{x}) = \mathcal{A}_i(b_i + \vec{w}_i \cdot \vec{x})$ for $i \in \{1, \dots, N_{\ell}\}$, which we can rearrange as a vector $\vec{y}$. The transformation mapping $\vec{x}$ into $\vec{y}$ can be directly modelled by means of matrix-vector multiplication, thus defining a new mathematical device, called \textit{linear (neural network) layer}:

$$\vec{y} = L(\vec{x}) = \mathcal{A}(\vec{b} + \mat{W}\vec{x})$$

with $\vec{y}, \vec{b} \in \mathbb{R}^m$, $\vec{x} \in \mathbb{R}^n$, $m,n \in \mathbb{N}_{\setminus0}$ and $\mat{W}$ an adequately-defined $m \times n$ matrix. In this case, $\mathcal{A}: \mathbb{R}^m \rightarrow \mathbb{R}^m$ is usually (but not always\footnote{See \cite{Xiao2020Enhancing} as a refreshing example of such kind.}!) just an elementwise application of the same scalar function.

By considering again a group of $N$ ordered (and distinct) \textit{layers} -- the $j^{\text{th}}$ of which taking as input the output of the $(j-1)^{\text{th}}$ -- we can finally define an \textit{N-layers deep fully-connected feedforward\footnote{\label{graphrepr}Given the absence of \textit{loops} in the (oriented) graph-based representation of the network. There, pre-activations are represented at the \textit{neuron} level one scalar per node; outgoing edges imply the application of the \textit{activation function} and multiplication with the scalar weight, whereas incoming edges, summation. Biases are encoded by the weight of additional edges with constant unit output. The set of all neurons with edges incoming into another (excluding biases) are called \textit{receptive field} of the latter.} artificial neural network} $\mathcal{N}_N$ as:

$$\mathcal{N}_N(\vec{x}) = L_N(L_{N-1}(\dots (L_2(L_1(\vec{x}))))) \text{ .}$$

The output (or \textit{post-activation}) of the $j^{\text{th}}$ layer (\textit{i.e.} $\vec{r_j} = L_j(\vec{r_{j-1}})$) -- or, according to a different convention, its \textit{pre-activation} (\textit{i.e.} $\vec{b_j} + \mat{W_j}\vec{r_{j-1}}$) -- is given the name of \textit{$j^{\text{th}}$-layer representation}. The same naming convention can be scaled down at the neuron level, or up to an entire network (by considering the ordered representations of all composing layers/neurons).

Additionally, all the weights and biases of an artificial neural network (considered as part of an ordered collection) are called -- as in the general statistical, \textit{ML}, or modelling setting -- \textit{parameters} of the model; on the other hand, all the additional characterising elements described by a quantitative choice (or even non-quantitative, \textit{lato sensu}) among many options -- not meant to be learned -- are called \textit{hyperparameters} (among which, \textit{e.g.}, the number of layers -- also called \textit{depth} of the network -- or the output size of each \textit{layer}).

\subsubsection{A whole \textit{bestiary} of networks and \textit{universal approximation}}

At this point, one may ask whether \textit{fully-connected feedforward artificial neural networks (FCNs)} are an adequate model to approximate the transformation of the inputs required by the task to be learned -- or if other kinds of similar models are available (and more appropriate).

As far as the latter question is concerned -- even if not strictly required for the prosecution of the present work -- many variations have been proposed since the first \textit{neural models}. Some of these operate at the level of \textit{single neurons} -- \textit{e.g.} by changing the way incoming inputs in the graph-representation (see: {\ref{graphrepr}}) of the network are handled, as in \textit{spiking neural networks}; others change the layer-wide behaviour -- \textit{e.g.} by structuring neuron connectivity and enforcing weight-sharing, as in \textit{convolutional neural networks} Others more change the connectivity of the network in ways that go beyond single layers (as in \textit{recurrent} or \textit{graph} \textit{neural networks}, where -- among other differences -- \textit{e.g.} \textit{feedbacks} are possible). Some of these cases -- and definitely others -- also require (or actually propose, as the only modification) a different \textit{learning algorithm}\footnote{The main \textit{learning algorithm} for deep neural models will be discussed in the following section.} to be employed \wrt already established choices.

Surely -- and beyond the simple variation of hyperparameters -- ever new artificial neural models are routinely developed within deep learning research and practice, by variously composing already established ones (\textit{i.e.} considering the outputs, or even more generally the representations or the parameters, of a network as the input of another), and by likewise adopting different training or inference strategies\footnote{Of this latter kind is main element of novelty of this thesis.}.

The adequacy of deep learning models to approximate a given map linking inputs and outputs of interest -- and specifically whether such approximation can be learned (and how!) from examples -- is a vast subject with rarely clear-cut answers, constantly evolving and attracting renovated research interest.

The main results so far -- known as \textit{universal approximation theorems} -- establish for a given \textit{artificial neural architecture}, seen as a family of algorithmically-generated functions parametrised by its weights and biases, their density within a function space of interest. Originally mostly focused on \textit{feedforward} architectures of fixed depth at the increase of width, spaces of continuous functions between Euclidean spaces, and the notion of \textit{density} induced by uniform convergence within compact sets -- over the years many extensions to such theorems have been proposed and proven (most notably in the case of fixed width and increasing depth, and for a variety of commonly used neural architectures such as \textit{convolutional neural networks} or those with a wide class of activation functions or subject to specific constraints).

In any case -- though the striking expressive power of deep artificial neural networks is undoubtedly confirmed by experimental results -- such theorems are almost always \textit{existence} ones, without providing a direct way of determining the (hyper)parameters actually approximating a given function within stated tolerance. Within the frame just described, the practical determination of such latter (hyper)parameters has largely been an empirical science -- relying upon clever modelling choices (\textit{e.g.} the choice of \textit{inductive biases} -- \textit{i.e.} the properties of specific architectures \wrt the input/output mapping produced), the development of ever more effective learning algorithms, and always experimental evaluation.

It will go beyond the scope of this thesis to discuss in further \textit{depth} or \textit{width} the current state of research in the field. Some cornerstone results are contained in \cite{HornikEtAl1989Approx}, \cite{Hornik1991Approx}, \cite{LinEtAl2018Approx}, \cite{Xu2018ApproxGNN}, and \cite{KidgerLyons2020Approx}.


\subsubsection{\textit{Actually learning} in Deep Learning}

We introduced \textit{deep learning} as a subset of \textit{machine learning} -- however, we have so far outlined just the \textit{modelling} part of it, and briefly mentioned some formal guarantees whose transfer into practice is hardly possible. How can we \textit{automatically} learn the the parameters of a deep learning model capable of approximately performing a given task (or at least give us the reasonable expectation it can) -- only from inputs (\textit{e.g.} in the case of unsupervised learning) or input/output pairs (\textit{e.g.} in the case of supervised learning)?

The answer is, in principle, exactly equivalent to that typical of \textit{traditional} numerical function approximation, or statistical model fitting.

Given a deep artificial neural network $\netw{N}$, we first define an adequate \textit{loss function} -- that should encode the degree of success with which the model is capable of solving the chosen task, and whose value generally depends on inputs and parameters (usually, but not exclusively, through $\netw{N}$) and, in the case of supervised learning, on the outputs. Then, we minimise the \textit{loss}, optimising \wrt the parameters, while evaluating it on the given \textit{training data}.

More precisely, calling $\vec{\theta}$ the collection of weights and biases for the entire model $\netw{N}$, and $\mathcal{L}_{\netw{N}}$ the chosen loss function, we seek the optimal parameters

$$\vec{\theta^*} \coloneqq \argmin_{\vec{\theta}} \mathcal{L}_{\netw{N}}(\vec{x}, \vec{y} | \vec{\theta}) \fullstop$$

Such simple formulation, however, hides one of the most relevant differences between deep learning and more traditional \textit{(approximate) model fitting} approaches: the parameter space can be extremely high-dimensional\footnote{\textit{E.g.}, among the largest \textit{deep neural models} to date, Google's \texttt{GLaM} -- see \cite{DuEtAl2022GLaM} -- boasts $>1$ trillion learnable parameters!}, and the \textit{loss landscape} -- \textit{i.e.} $\mathcal{L}_{\netw{N}}(\vec{\theta})$ -- highly nonconvex, \textit{rugged}, with abundant of local minima and/or saddle points\footnote{For an impactful visualisation, see \cite{LiEtAl2018VisualizingLL}.}. This rules out any direct global optimisation approach, for problems beyond \textit{toy examples}.

Though not necessarily the \textit{only} choice -- the almost-totality of learning algorithms for deep neural models is based upon \textit{gradient descent iterations}, \textit{i.e.} the approximation $\vec{\theta^*} \approx \vec{\theta_{\hat{i}}}$ for a sufficiently large $\hat{i}$, and the following iteration:

\begin{equation} \label{eq:1}
    \vec{\theta_{i+1}} \leftarrow  \vec{\theta_{i}} - \lambda\vec{g_i}
\end{equation}

with $\lambda \in \mathbb{R}_+$ (\textit{learning rate}) and $\vec{g_i} \coloneqq \left.\frac{\ddiff{\mathcal{L}_{\netw{N}}(\vec{\theta})}}{\ddiff{\vec{\theta}}} \right|_{\vec{\theta_{i}}}$ (\textit{local gradient}).

The exact dependence of $\mathcal{L}_{\netw{N}}$ from $\vec{x}$ directly relates to the specific choice of an \textit{aggregation scheme} for the elementwise $\vec{\theta_{i}}$-gradients computed for each datapoint available.

Let us suppose the usual \textit{supervised learning} setting, with $\{(\vec{x_i}, \vec{y_i})\}_{i = \{1,2,\dots,N\}}$ as the \textit{training dataset}. $\vec{g_i}$ iterations can then be defined as\footnote{An equivalent formulation is also possible -- with summation replaced by averaging of the gradients -- implying a rescaling of the learning rate.}

$$\vec{g_i} \coloneqq \sum_{k \in \mathcal{B}_j \subseteq \{1,2,\dots,N\}}{\left.\frac{\ddiff{\mathcal{L}_{\netw{N}}(\vec{x_k}, \vec{y_k}|\vec{\theta})}}{\ddiff{\vec{\theta}}} \right|_{\vec{\theta_{i}}}}$$

with simultaneous updates of $i$ and $j$ (\textit{i.e.} for two consecutive iterations, summation is performed on different $\mathcal{B}_j$s), and $\bigcup_j\mathcal{B}_j$ a partition of the (generally shuffled) $\{1,2,\dots,N\}$.

The various $\{(\vec{x_i}, \vec{y_i})\}_{i \in \mathcal{B}_j}$s are called \textit{(mini)batches} of the dataset, and their size (fixed, except eventually for last one) $B$ (\textit{batch size}) is such that $\#\text{batches} = \lceil \frac{N}{B}\rceil$ -- whereas the corresponding $\vec{g_i}$s are called \textit{noisy local gradients} in case $B \neq N$ (as it is, indeed, a \textit{noisy estimate} of the actual \textit{local gradient}).

The choice of $B$ and $\lambda$ (which are additional \textit{hyperparameters}, of the learning algorithm this time) can influence the convergence of the iterations. While it is true that \textit{noisy gradient} iterations approximate those of true gradients as $i$ grows, the choice of a larger $B$ (up to the limit $B=N$, called \textit{(full) batch gradient descent}) reduces the variance of the estimate at the cost of increased susceptibility to convergence toward local minima. On the other hand, a decrease in $B$ (down to $B=1$, \textit{stochastic gradient descent}\footnote{Note, however, that such name is commonly used in practice to describe the whole family of these optimisation methods, regardless of the choice of $B$.}) favours convergence toward the global minimum at the price of increased variance and number of iterations required to reach it. The number of times the training set (in the form of batches, eventually) is entirely used during training is called \textit{number of epochs}. Ultimately, the choice of \textit{batch size} in modern day boils down to the compromise\footnote{See, \textit{e.g.} \cite{MastersLuschi2018RevisitingSB} for an unusual take on the subject,  \href{https://twitter.com/ylecun/status/989610208497360896?lang=en}{endorsed} by deep learning pioneer and expert Yann LeCun.} between \textit{regularisation} of the optimisation problem and (reduction in) the number of iterations potentially required for proper convergence -- provided in any case sufficient memory to store the whole minibatch\footnote{While it is always possible to resort to \textit{gradient accumulation} to counter memory starvation (see \href{https://discuss.pytorch.org/t/why-do-we-need-to-set-the-gradients-manually-to-zero-in-pytorch/4903/20}{this forum comment by one of the PyTorch developers} as an explanatory example), a memory-fittable alternative has to be preferred, due to the non-summability of \textit{batch normalisation} statistics. The analysis of such regularisation technique will be discussed right next.}.

Finally, it must be stressed that \textit{optimisation} plays a crucial role in the \textit{learning algorithm} of a deep neural architecture: for that reason, many variations of the prototypical iteration described in (\ref{eq:1}) have been proposed. Chiefly -- they augment such iteration with additional \textit{iteration variables} (and subsequent additive terms to the gradient) in order to better exploit \textit{greater-than-first} order information about the \textit{loss landscape} and thus provide faster and/or more accurate convergence to the \textit{true} global minimum. A thorough description of such proposals is again out of the scope of this work: however, the most common device used in this context must be mentioned. It is the case of \textit{exponentially-weighted averages} of the gradient. As an example, the nowadays ubiquitous \textit{stochastic gradient descent with momentum} variation proposes a simultaneous iteration of the type:

$$\vec{m_{i+1}} \leftarrow  \vec{\theta_{i+1}} - \vec{\theta_{i}}$$
$$\vec{\theta_{i+1}} \leftarrow  \vec{\theta_{i}} - \lambda\vec{g_i} + \beta\vec{m_i}$$

with $\beta \in (0,1]$. Such addition -- mimicking the \textit{momentum}, indeed, of a body moving subject to a potential $\mathcal{L}$ -- \textit{e.g.} both provides noise attenuation for the gradient estimates and ameliorates the convergence towards global minimum in the case of highly unbalanced (in absolute value) gradient components across dimensions. Additionally, \texttt{Adam}\footnote{See \cite{KingmaBa2015Adam}. Also, \texttt{Adam} is the basis for the main optimiser used in the learning protocol proposed by this work -- \texttt{RAdam} (see \cite{LiuEtAl2020OnTheVariance}). The improvement upon \texttt{Adam} consists in an accurate analytical \textit{de-biasing} of the \textit{adaptive learning rate} variance, especially in the first few iterations of the algorithm, similarly to the previously-known \textit{warmup} heuristic.} is one more among the most relevant and used variations -- which further regularise learning via modulation of the \textit{effective learning rate} in accordance to better estimates of local curvature.

\subsubsection{\textit{Overparametrisation} and regularisation strategies}

As we have anticipated earlier, the number of parameters of a deep neural model can be even extremely large -- and usually purposefully so: excessive parameter parsimony may in fact artificially cap model expressiveness beyond the (usually unknown beforehand) requirements of the task to be learned, and render optimisation by gradient descent harder\footnote{Such latter statement has been heuristically explained by Terrence Sejnowski as follows: the probability that no direction of the \textit{local gradient} points towards a pre-set point -- in this case, the global optimum -- decreases as the dimension of the parameters space increases, even for a randomly-picked \textit{local gradient} vector.}.

Still, the deliberate modelling choice of employing a number of parameters (much) larger than reasonably estimable -- so-called \textit{overparametrisation} -- does come at the cost of increased risk of overfitting\footnote{The loss-minimisation-driven adaptation of the model to the training dataset at the point of losing generalisation ability -- \textit{i.e.} the inability to learn the abstract task beyond the specific exemplifying data.} and unmanageable complexity. To counteract such downsides -- as in \textit{traditional} statistical practice -- \textit{regularisation strategies} have been developed, some of them specific to deep learning, which will be discussed next.

\paragraphnl{Weight decay}
First of all, one could apply -- to the optimisation problem of finding $\vec{\theta^*}$ -- the same \textit{penalisation} regularisation techniques typical of \textit{traditional} statistics; in particular, $L_2$-penalised fitting (\textit{à la} ridge regression). Such approach is called -- within the deep learning community -- \textit{weight decay}\footnote{See \cite{KroghPalmer1991WeightDecay}.}; it is indeed possible, for iterations of the type described in (\ref{eq:1}) to express the $\mathcal{L}_{\text{ridge}} \leftarrow \mathcal{L} + \frac{\gamma}{2}||\vec{\theta}||_2$ regularisation as a modified iteration, \textit{i.e.}

$$\vec{\theta_{i+1}} \leftarrow  \vec{\theta_{i}} - \lambda\vec{g_i} - \lambda\gamma\vec{\theta_i}\fullstop$$

The two routes, however, become different in the cases -- outlined above -- where exponentially-weighted averaging, too, comes into play at the optimisation step by means of a more sophisticated optimisation scheme. This previously unnoticed detail has since then produced, once again, different variations of previously known optimisers\footnote{See, \textit{e.g.}, the most relevant optimiser of this kind, \texttt{AdamW} -- described in \cite{LoshchilovHutter2018AdamW}.} -- with an empirical general preference for proper \textit{weight decay}, but no clear-cut conclusion (and a relatively modest effect, compared with other \textit{tricks}).

\paragraphnl{\textit{Dropout}}
Another\footnote{See \cite{SirvastavaEtAl2014Dropout}.} -- \textit{deep-learning-specific} -- technique directly attempts model fitting with a stochastically-selected subset of parameters: for each batch of training data, some randomly-sampled neuron representations (in a given pre-set proportion, layerwise) are forced to zero, resulting in the corresponding weights and biases to be unmodified by the learning iteration. At \textit{inference time}, no constraint is imposed, but each \textit{weight} is further weighted by the corresponding complementary probability of \textit{training time} zeroing.

Such technique -- and its more advanced variations\footnote{See \textit{e.g.} \cite{BolukiEtAl2020BernoulliDropout}.} -- have demonstrated remarkable success in improving generalisation, by taking into account in a less intertwined fashion the various different \textit{pathways} activated through the model by a given input example -- thus increasing the overall parameter efficiency of the network. This comes at the cost of a generally less \textit{sparse} and more \textit{distributed} representation -- which may not always be an intended goal of the training.

\paragraphnl{\textit{Batch Normalisation}}
Though not properly considered a \textit{regularisation technique}, \textit{batch normalisation} has a twofold effect: speeding up and improving converge to the \textit{true} global minimum for practically any optimiser, and reducing the \textit{noisiness} of batched input data -- by normalising each datapoint coordinate within its respective batch, and further scaling and shifting it according to further learnable parameters.

Initially considered to due its efficacy to a reduction in so-called \textit{internal covariate shift}\footnote{See the original paper, \textit{i.e.} \cite{IoffeSzegedy2015BatchNorm}.}, it has been later established\footnote{\cite{SanturkarEtAl2018BatchNorm}.} that its main contribution is a net smoothing effect on the loss landscape.

\paragraphnl{Learning rate scheduling}
Finally, in the same family of \textit{improper regularisation techniques} as \textit{batch normalisation}, there is \textit{learning rate scheduling}: the adaptation of \textit{epoch-} (or, more generally \textit{batch-}) \textit{specific} learning rate to some pre-set or adaptive schedule. Such technique, originally developed for \textit{non-learning-rate-adaptive} optimisers, further speeds up convergence -- while simultaneously avoiding that excessively large steps during the initial iterations of the algorithm steer convergence away from the global minimum, and during the later stages prevent convergence to narrower, but optimal, basins. Many scheduling schemes -- with varying levels of theoretical justification and/or empirical vetting, but no definitive evidence across all possible application scenarios -- have been proposed, and their actual choice often depends on habit, compromise between improvement and additional hyperparameters to tune, or \textit{brute force} \textit{trial and error}.

\subsubsection{\textit{Algorithmic differentiation} and the \textit{backpropagation} algorithm}

We are left -- at this point -- with one last, but crucial, question: how does all the \textit{machinery} outlined until now work in practice, at the implementation level? In fact, until now, we have neither talked about how the terms $\vec{g_i}$ in (\ref{eq:1}) are computed from the mere datapoint-wise knowledge of $\mathcal{L}(\vec{x}|\vec{\theta})$, nor we have put specific constraints on the differentiability (or even continuity!) of $\mathcal{L}$: and in fact such constraints are mostly unnecessary in practice\footnote{Indeed, heuristically -- for any reasonably modern \textit{automatic differentiation} framework, such as \href{https://pytorch.org/}{PyTorch}, see \cite{PaszkeEtAl2019PyTorch} -- it is strictly sufficient that $\mathcal{L}(\vec{\theta})$ and $\mathcal{L}(\vec{x})$ are piecewise \textit{dual-number}-differentiable within open sets, and such \textit{pieces} definable by algorithmic branching and/or recursion. \textit{I.e.} discontinuous activation functions are allowed, and so are selection function such as in-place sorting.}.

The core ingredient seamless allowing such kind of computations is (collectively) called \textit{automatic} or \textit{algorithmic} \textit{differentiation} -- and refers to a general algorithmic strategy to determine exact pointwise evaluations\footnote{\textit{I.e.} without resorting to \textit{numeric approximation} of derivatives, and neither computing them symbolically such as in \textit{computer algebra systems} or \textit{e.g.} \href{https://www.sympy.org/}{\texttt{SymPy}}.} of derivatives for (practically) any piece of \textit{legal} code. The underlying principles and technicalities powering such approaches go far beyond the scope of this work. As a \textit{proof-of-concept} justification, an intuitive mechanism allowing similar flexibility (though rarely used in modern \textit{AutoDiff} frameworks) -- that of \textit{dual numbers} -- is thoroughly described in \cite{Fischer1998DualNumbers}. An accurate and comprehensive survey of \textit{automatic differentiation} methods for machine (and \textit{deep}, indeed) learning is available in \cite{BaydinEtAl2018AutoDiff}.

Finally, in order to ensure computational efficiency for the whole process, the \textit{error backpropagation} algorithm is employed -- guaranteeing that the term $\vec{g_i}$ is computed linearly in the number of parameters \wrt the number of atomic differentiation operations. This comes possible thanks to the graph-based representation of a neural architecture, the application of the \textit{chain rule of differentiation} (by noting that in a \textit{multilayer} architecture the derivatives of representations at layer $i$ only depend from at most all the representations of layer $i-1$), and the \textit{memoisation} of such derivatives during the computations required by those of the innermost layer. Such approach to the computation of the entire $\frac{\partial\mathcal{L}(\vec{\theta})}{\partial\vec{\theta}}$ Jacobian from a reverse network-graph traversal (and the relevant related data structures built upon it) is called \textit{reverse-mode automatic differentiation} and powers the large majority of modern deep learning libraries.

\subsubsection{Advanced topics in \textit{deep learning}}

\subsection{\textit{Adversarial Robustness} (and lack thereof)}
